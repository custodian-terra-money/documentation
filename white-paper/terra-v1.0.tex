%% LyX 2.3.1-1 created this file.  For more info, see http://www.lyx.org/.
%% Do not edit unless you really know what you are doing.
\documentclass[11pt]{article}
\usepackage[LGR,T1]{fontenc}
\usepackage[latin9]{inputenc}
\usepackage{geometry}
\geometry{verbose,tmargin=1in,bmargin=1in,lmargin=1in,rmargin=1in}
\usepackage{color}
\usepackage{amsmath}
\usepackage{amssymb}
\usepackage{graphicx}
\usepackage{pgfplots}
\usepackage{setspace}
\doublespacing
\usepackage[unicode=true,
 bookmarks=false,
 breaklinks=false,pdfborder={0 0 1},backref=section,colorlinks=true]
 {hyperref}
\hypersetup{pdftitle={Terra Money White Paper},
 linkcolor=black,citecolor=blue,filecolor=magenta,urlcolor=blue}

\makeatletter

%%%%%%%%%%%%%%%%%%%%%%%%%%%%%% LyX specific LaTeX commands.
\DeclareRobustCommand{\greektext}{%
  \fontencoding{LGR}\selectfont\def\encodingdefault{LGR}}
\DeclareRobustCommand{\textgreek}[1]{\leavevmode{\greektext #1}}
\ProvideTextCommand{\~}{LGR}[1]{\char126#1}


%%%%%%%%%%%%%%%%%%%%%%%%%%%%%% User specified LaTeX commands.
\usepackage{color,titling,titlesec}
\usepackage{eurosym}
\usepackage{harvard}\usepackage{amsfonts}
\usepackage{graphicx,pdflscape}
\usepackage{chicago}\setcounter{MaxMatrixCols}{30}
\usepackage[bottom]{footmisc}

% This shrinks the space before and after display formulas
\usepackage{etoolbox}
\apptocmd\normalsize{%
\abovedisplayskip=5pt
%\abovedisplayshortskip=6pt % plus 3pt
 \belowdisplayskip=6pt
 %\belowdisplayshortskip=7pt plus 3pt
}{}{}

% slim white space before title, section/subsection headers
\setlength{\droptitle}{-.75in}
\titlespacing*{\section}{0pt}{.2cm}{0pt}
\titlespacing*{\subsection}{0pt}{.2cm}{0pt}

\makeatother

\begin{document}
\title{Terra Money:\\
Stability and Adoption}
\author{Evan Kereiakes, Do Kwon,
Marco Di Maggio, Nicholas Platias}
\date{February 2019}
\maketitle
\begin{center}
{\large{}\vspace{-1.5cm}
}{\large\par}
\par\end{center}
\begin{abstract}
\begin{singlespace}
While many see the benefits of a price-stable cryptocurrency that
combines the best of both fiat and Bitcoin, not many have a clear
plan to get such a currency adopted. Since the value of a currency
as medium of exchange is mainly driven by its network effects, a successful
new digital currency needs to maximize adoption in order to become
useful. We propose a cryptocurrency, Terra, which is both price-stable
and growth-driven. It achieves price-stability via an elastic money
supply enabled by countercyclical mining incentives and transaction
fees. It also uses seigniorage created by its minting operations as
transaction stimulus, thereby facilitating adoption. There is demand
for a decentralized, price-stable money protocol in both fiat and
blockchain economies. If such a protocol succeeds, then it will have
a significant impact as the best use case for cryptocurrencies.
\end{singlespace}
\end{abstract}
\thispagestyle{empty}

\newpage\setcounter{page}{1}

\section{Introduction}

The price-volatility of cryptocurrencies is a well-studied problem
by both academics and market observers (see for instance, Liu and
Tsyvinski, 2018, Makarov and Schoar, 2018). Most cryptocurrencies,
including Bitcoin, have a predetermined issuance schedule that, together
with a strong speculative demand, contributes to wild fluctuations
in price. Bitcoin\textquoteright s extreme price volatility is a major
roadblock towards its adoption as a medium of exchange or store of
value. Intuitively, nobody wants to pay with a currency that has the
potential to double in value in a few days, or wants to be paid in
the currency if its value can significantly decline before the transaction
is settled. The problems are aggravated when the transaction requires
more time, e.g. for deferred payments such as mortgages or employment
contracts, as volatility would severely disadvantage one side of the
contract, making the usage of existing digital currencies in these
settings prohibitively expensive. 

At the core of how the Terra Protocol
solves these issues is the idea that a cryptocurrency with an elastic
monetary policy would stabilize its price, retaining all the censorship
resistance of Bitcoin, and making it viable for use in everyday transactions. However, price-stability is not sufficient to get a new currency widely adopted. Currencies are inherently characterized by strong network
effects: a customer is unlikely to switch over to a new currency unless
a critical mass of merchants are ready to accept it, but at the same
time, merchants have no reason to invest resources and educate staff
to accept a new currency unless there is significant customer demand
for it. For this reason, Bitcoin\textquoteright s adoption in the
payments space has been limited to small businesses whose owners are
personally invested in cryptocurrencies. Our belief is that while
an elastic monetary policy is the solution to the stability problem,
an efficient fiscal policy can drive adoption. Then, the Terra Protocol
also offers strong incentives for users to join the network with an
efficient fiscal spending regime, managed by a Treasury, where multiple
stimulus programs compete for financing. That is, proposals from community
participants will be vetted by the rest of the ecosystem and, when
approved, they will be financed with the objective to increase adoption
and expand the potential use cases. 

The Terra Protocol with its balance
between fostering stability and adoption represents a meaningful complement
to fiat currencies as means of payment, and store of value. The rest
of the paper is organized as follows. We first discuss the protocol
and how stability is achieved and maintained, through the calibration
of miners\textquoteright{} demand and the use of the native mining
Luna token. We then dig deeper in how countercyclical incentives and
fees are adopted to smooth fluctuations. Second, we discuss how Terra\textquoteright s
fiscal policy can be used as an efficient stimulus to drive adoption. 

\section{Multi-fiat peg monetary policy }

A stable-coin mechanism must answer three key questions: 
\begin{itemize}
\item \textbf{How is price-stability defined?} Stability is a relative concept;
which asset should a stable-coin be pegged to in order to appeal to
the broadest possible audience? 
\item \textbf{How is price-stability measured?} Coin price is exogenous
to the Terra blockchain, and an efficient, corruption-resistant price
feed is necessary for the system to function properly. 
\item \textbf{How is price-stability achieved?} When coin price has deviated
from the target, the system needs a way to apply pressures to the
market to bring price back to the target. 
\end{itemize}
This section will specify Terra\textquoteright s answers to the above
questions in detail. 

\subsection{Defining stability against regional fiat currencies }

The existential objective of a stable-coin is to retain its purchasing
power. Given that most goods and services are consumed domestically,
it is important to create crypto-currencies that track the value of
local fiat currencies. Though the US Dollar dominates international
trade and forex operations, to the average consumer the dollar exhibits
unacceptable volatility against her choice unit of account. 

Recognizing strong regionalities in money, Terra aims to be a family of cryptocurrencies that are each pegged to the world's major currencies. Close to genesis, the protocol will issue Terra currencies pegged to USD, EUR, CNY, JPY, GBP, KRW, and the IMF SDR. Over time, more currencies will be added to the list by user voting. TerraSDR will be the flagship currency of this family, given that it exhibits the lowest volatility against any one fiat currency (Kereiakes, 2018). TerraSDR be the currency in which transaction fees, miner rewards and stimulus grants will be denominated.

It is important, however, for Terra currencies to have access to shared liquidity. For this reason, the system supports atomic swaps among Terra currencies at their market exchange rates. A user can swap TerraKRW for TerraUSD instantly at the effective KRW/USD exchange rate. This allows all Terra currencies to share liquidity and macroeconomic fluctuations; a fall in demand by one currency can quickly be absorbed by the others. We can therefore reason about the stability of Terra currencies in a group; we will be referring to Terra loosely as a single currency for the remainder of this paper. As Terra's ecosystem adds more currencies, its atomic swap functionality can be an instant solution to cross border transactions, international trade settlements and usurious capital controls. 




\subsection{Measuring stability with miner oracles }

Since the price of Terra currencies in secondary markets is exogenous to the blockchain, the system must rely on a decentralized price oracle to estimate the true exchange rate. We define the mechanism for the price oracle as the following: 

\begin{itemize}
\item For any Terra sub-currency in the set of currencies C = {TerraKRW, TerraUSD, TerraSDR... } miners submit a vote for what they believe to be the current exchange rate in the target fiat asset. 
\item Every n blocks the vote is tallied by taking the weighted medians as the true rates.   
\item Some amount of Terra is rewarded to those who voted within 1 standard deviation of the elected median. Those who voted outside may be punished via slashing of their stakes. The ratio of those that are punished and rewarded may be calibrated by the system every vote to ensure that a sufficiently large portion of the miners vote.
\end{itemize}

Several issues have been raised in implementing decentralized oracles, but chief among them is the possibility for voters to profit by coordinating on a false price vote. Limiting the vote to a specific subset of users with strong vested interest in the system, the miners, can vastly decrease the odds of such a coordination. A successful coordination event on the price oracle would result in a much higher loss in the value of the miner stakes than any potential gains, as Luna stakes are time-bound to the system.  

The oracle can also play a role in adding and deprecating Terra currencies. The protocol may start supporting a new Terra currency when oracle votes for it satisfies a submission threshold. Similarly, the failure to receive a sufficient number of oracle votes for several periods could trigger the deprecation of a Terra currency. 


\subsection{Achieving stability through countercyclical mining }

Once the system has detected that the price of a Terra currency has
deviated from its peg, it must apply pressures to normalize the price.
Like any other market, the Terra money market follows simple rules
of supply and demand for a pegged currency. That is: 
\begin{itemize}
\item Contracting money supply, all conditions held equal, will result in
higher relative price levels. That is, when price levels are falling
below the target, reducing money supply sufficiently will return price
levels to normalcy. 
\item Expanding money supply, all conditions held equal, will result in
lower relative price levels. That is, when price levels are rising
above the target, increasing money supply sufficiently will return
price levels to normalcy. 
\end{itemize}
Of course, contracting the supply of money isn\textquoteright t free;
like any other asset, money needs to be bought from the market. Central
banks and governments shoulder contractionary costs for pegged fiat
systems through a variety of mechanisms including intervention, the
issuance of bonds and short-term instruments thus incurring costs
of interest, and hiking of money market rates and reserve ratio requirements
thus losing revenue. Put in a different way, central banks and governments
absorb the volatility of the pegged currencies they issue. 

Analogously, Terra miners absorb volatility in Terra supply. 
\begin{itemize}
\item \textbf{In the short term, miners absorb Terra contraction costs}
through mining power dilution. During a contraction, the system mints
and auctions more mining power to buy back and burn Terra. This effectively
contracts the supply of Terra until its price has returned to the
peg, and temporarily results in mining power dilution. 
\item \textbf{In the mid to long term, miners are compensated with increased
mining rewards}. First, the system continues to buy back mining power
until a fixed target supply is reached, thereby creating long-run
dependability in available mining power. Second, the system increases
mining rewards, which will be explained in more detail in a later
section. 
\end{itemize}
In summary, miners bear the costs of Terra volatility in the short
term, while being compensated for it in the long-term. Compared to
ordinary users, miners have a long-term vested interest in the stability
of the system, with invested infrastructure, trained staff and business
models with high switching cost. The remainder of this section will
discuss how the system forwards short-term volatility and creates
long-term incentives for Terra miners. 

\subsection{Miners absorb short-term Terra volatility }

The Terra Protocol runs on a Proof of Stake (PoS) blockchain, where
miners need to stake a native cryptocurrency Luna to mine Terra transactions.
At every block period, the protocol elects among the set of staked
miners a block producer, which is entrusted with the work required
to produce the next block by aggregating transactions, achieving consensus
among miners, and ensuring that messages are distributed properly
in a short timeframe with high fault tolerance. 

The block producer election is weighted by the size of the active
miner\textquoteright s Luna stake. Therefore, \textbf{Luna represents
mining power in the Terra network.} Similar to how a Bitcoin miner\textquoteright s
hash power represents a pro-rata odds of generating Bitcoin blocks,
the Luna stake represents pro-rata odds of generating Terra blocks. 

Luna also serves as the most immediate defense against Terra price
fluctuations. The system uses Luna to make the price for Terra by
agreeing to be counter-party to anyone looking to swap Terra and Luna
at Terra's target exchange rate. More concretely: 
\begin{itemize}
\item When TerraSDR's price < 1 SDR, users and arbitragers can send 1 TerraSDR
to the system and receive 1 SDR's worth of Luna. 
\item When TerraSDR's price > 1 SDR, users and arbitragers can send 1 SDR's
worth of Luna to the system and receive 1 TerraSDR. 
\end{itemize}

The system's willingness to respect the target exchange rate irrespective
of market conditions keeps the market exchange rate of Terra at a
tight band around the target exchange rate. An arbitrager can benefit when 1 TerraSDR = 0.9 SDR by trading TerraSDR for 1 SDR's worth of Luna from the system, as opposed to 0.9 SDR's worth of assets she could get from the open market. Similarly, she can also benefit when 1 TerraSDR = 1.1 SDR by trading in 1 SDR’s worth of Luna to the system to get 1.1 SDR’s worth of TerraSDR, once again beating the price of the open market. 

The system finances Terra price making via Luna: 
\begin{itemize}
\item To buy 1 TerraSDR, the protocol mints and sells Luna worth 1 SDR 
\item By selling 1 TerraSDR, the protocol earns Luna worth 1 SDR 
\end{itemize}
As Luna is minted to match Terra offers, volatility is moved from
Terra price to Luna supply. If unmitigated, this Luna dilution presents
a problem for miners; their Luna stakes are worth a smaller portion
of total available mining power post-contraction. The system
burns a portion of the Luna it has earned during expansions until
Luna supply has reached its 1 billion equilibrium issuance. Therefore,
Luna can have steady demand as a token with pro-rata rights to Terra
mining over the long term. The next section discusses how the system
offers countercyclical mining incentives to keep the market for mining
and demand for Luna long-term stable through volatile macroeconomic
cycles. 

\subsection{Countercyclical Mining Rewards }

Our objective is to counteract fluctuations in the value of mining
Terra by calibrating mining rewards to be countercyclical. The main
intuition behind a countercyclical policy is that it attempts to counteract
economic cycles by increasing mining rewards during recessions and
decreasing mining rewards during booms. The protocol has two levers
at its disposal to calibrate mining rewards: transaction fees, and
the proportion of seigniorage that gets allocated to miners. 
\begin{itemize}
\item Transaction fees: The protocol levies a small fee from every Terra
transaction to reward miners. Fees default to 0.1\% but may vary over
time to smooth out mining rewards. If mining rewards are declining,
an increase in fees can reverse that trend. Conversely, high mining
rewards give the protocol leeway to bring fees down. Fees are capped
at 2\%, so they are restricted to a range that does not exceed the
fees paid to traditional payment processors. 
\item Seigniorage: Users can mint Terra by paying the system Luna. This
Luna earned by the system is seigniorage, the value of newly minted
currency minus the cost of issuance. The system burns a portion of
seigniorage, which makes mining power scarcer and reduces mining competition.
The remaining portion of seigniorage goes to the Treasury to finance
fiscal stimulus. The system can calibrate the allocation of seigniorage
between those two destinations to impact mining reward profiles. 
\end{itemize}
We use two key macroeconomic indicators as inputs for controlling
mining rewards: money supply and transaction volume. Those two indicators
are important signals for the performance of the economy. Looking
at the equation for mining rewards: when the economy underperforms
relative to recent history (money supply and transaction volume have
decreased) a higher proportion of seigniorage is allocated to mining
rewards, and transaction fees increase; conversely, when the economy
outperforms recent history (money supply and transaction volume have
increased) a lower proportion of seigniorage is allocated to mining
rewards, and transaction fees decrease. 

Phrasing the above more formally, both the proportion of seigniorage
that gets allocated to mining rewards $w_t$ and the adjustment in transaction
fees $f_t$ are controlled over time as follows:

\[
w_{t+1}=w_{t}+\beta\left(\frac{M_{t}}{M_{t}^{*}}-1\right)+\gamma\left(\frac{TV_{t}}{TV_{t}^{*}}-1\right)
\]

\[
f_{t+1}=f_{t}+\kappa\left(\frac{M_{t}}{M_{t}^{*}}-1\right)+\lambda\left(\frac{TV_{t}}{TV_{t}^{*}}-1\right)
\]

In the above $M_{t}$ is Terra money supply at time $t$, $M_{t}^{*}$
is the historical moving average of money supply over the previous
quarter, $TV_{t}$ is Terra transaction volume at time $t$ and $TV_{t}^{*}$
is correspondingly the historical moving average of transaction volume
over the previous quarter. The parameters \textgreek{b}, \textgreek{g},
\textgreek{k} and \textgreek{l} are all negative real numbers in the
range {[}-1, 0) and will be calibrated to produce responses that are
gradual but effective. Indicative values that have worked well in
our simulations are between -0.5 and -1 for \textgreek{b}, \textgreek{g}
and between -0.005 and -0.01 for \textgreek{k}, \textgreek{l} respectively.
Both $w_t$ and $f_t$ are restricted within the range imposed by the protocol
(between 10\% and 90\% for $w_t$, and between 0.1\% and 2\% for $f_t$). 

To see how this might work in practice: say that money supply is 10\%
higher than quarterly average and transaction volume is 20\% higher
than quarterly average. Let \textgreek{b}, \textgreek{g} be -0.5 and
\textgreek{k}, \textgreek{l} be -0.01 respectively. The seigniorage
allocation weight to mining rewards would decrease by 15\% and transaction
fees would decrease by 0.3\% (both on an absolute basis). These are
reasonable adjustments that allocate proportionally more capital to
the Treasury and ease the fee burden on users in response to strong
performance of the economy.
\begin{center}
\includegraphics[scale=0.42]{graph_tvgap}
\par\end{center}

Alternatively, as the graph shows, when the gap widens, i.e. as the
economy shrinks, the fees will increase from the example starting
point of 0.1\% to the maximum of 2\% that will still make Terra more
convenient than Visa and Mastercard. 

The rule we have outlined for making mining rewards countercyclical
is simple, intuitive and easily programmable. It takes inspiration
from Taylor\textquoteright s Rule (Taylor, 1993), utilized by monetary
authorities banks to help frame the level of nominal interest rates
as a function of inflation and output. Similarly, exactly as a central
bank, the protocol observes the health of the economy, in our case
the money supply and transaction volume, and adjusts its main levers
to ensure the sustainability of the economy.


\section{Terra Platform}
Smart contracts have enormous potential, but their use cases are limited by the price volatility of their underlying currency. The canonical function of a smart contract is to hold an escrow of tokens to be distributed when some set of conditions are triggered. Such a scheme is quite simply a futures contract, where all involved parties are forced to speculate on the price movement of the funds held by the contract. Price volatility makes smart contracts unusable for most mainstream financial applications, as most users are used to value determinate payouts in insurance, credit, mortgage, and payroll.  

The introduction of a stable dApp platform will allow smart contracts to mature into a useful infrastructure for mainstream businesses. Though most dApps today issue native tokens with custom token economics, for vast majority of cases such tokens have limited use cases and fragments the overall user experience, as users today needs to sell tokens A and buy tokens B and C to interact with dApps. Instead, the Terra Platform will be oriented to building financial applications that use Terra as their underlying currency.  

Terra Platform DApps will help to drive growth and stabilize the Terra by diversifying its use cases. The protocol may therefore subsidize the growth of the more successful applications through its growth-driven fiscal policy, and we talk about this in the next section.


\section{Growth-driven fiscal policy}

National governments use expansionary fiscal spending with the objective
of stimulating growth. On the balance, the hope of fiscal spending
is that the economic activity instigated by the original spending
results in a feedback loop that grows the economy more than the amount
of money spent in the initial stimulus. This concept is captured by
the spending multiplier \textemdash{} how many dollars of economic
activity does one dollar of fiscal spending generate? The spending
multiplier increases with the marginal propensity to consume, meaning
that the effectiveness of the expansionary stimulus is directly related
to how likely economic agents are to increase their spending. 

In a previous section, we discussed how Terra seigniorage is directed
to both miner rewards and the Treasury. At this point, it is worth
describing how exactly the Treasury implements Terra's fiscal spending
policy, with its core mandate being stimulating Terra's growth while
ensuring its stability. In this manner, Terra achieves greater efficiency
by returning seigniorage not allocated for stability back to its users.

The Treasury's main focus is the allocation of resources derived from
seigniorage to decentralized application (dApp). To receive seigniorage
from the Treasury, a dApp needs to register for consideration as an
entity that operates on the Terra network. dApps are eligible for
funding depending on their economic activity and use of funding. A
dApp registers a wallet with the network that is used to track economic
activity. Transactions that go through the wallet count towards the
dApp's transaction volume.

The funding procedure for a dApp works as follows: 
\begin{itemize}
\item A dApp applies for an account with the Treasury; the application includes
metadata such as the Title, a url leading to a detailed page regarding
the use of funding, the wallet address of the applicant, as well as
auditing and governance procedures. 
\item At regular voting intervals, Luna validators vote to accept or reject
new dApp applications for Treasury accounts. The net number of votes
(yes votes minus no votes) needs to exceed 1/3 of total available
validator power for an application to be accepted.
\item Luna validators may only exercise control over which dApps can open
accounts with the Treasury. The funding itself is determined programmatically
for each funding period by a weight that is assigned to each dApp.
This allows the Treasury to prioritize dApps that earn the most funding. 
\item At each voting session, Luna validators have the right to request
that a dApp be blacklisted, for example because it behaves dishonestly
or fails to account for its use of Treasury funds. Again, the net
number of votes (yes votes minus no votes) needs to exceed 1/3 of
total available validator power for the blacklist to be enforced.
A blacklisted dApp loses access to its Treasury account and is no
longer eligible for funding.
\end{itemize}
The motivation behind assigning funding weights to dApps is to maximize
the impact of the stimulus on the economy by rewarding the dApps that
are more likely to have a positive effect on the economy. The Treasury
uses two criteria for allocating spending: (1) \textbf{robust economic
activity} and (2) \textbf{efficient use of funding}. dApps with a
strong track record of adoption receive support for their continued
success, and dApps that have grown relative to their funding are rewarded
with more seigniorage, as they have a successful track record of efficiently
using their resources. 

Those two criteria are combined into a single weight which determines
the relative funding that dApps receive from the aggregate funding
pool. For instance, a dApp with a weight of 2 would receive twice
the amount of funding of a dApp with a weight of 1.

We lay out the funding weight equation, followed by a detailed explanation
of all the parts: For a time period t, let TV\_t be a dApp's transaction
volume and F\_t be the Treasury funding received. Then we determine
the funding weight $w_{t}$ for the period as follows:

\[
w_{t}=\left(1-\lambda\right)TV_{t}^{*}+\lambda\frac{\Delta TV_{t}^{*}}{F_{t-1}^{*}}
\]

The notation {*} denotes a moving average, so TV{*}\_t would be the
moving average of transaction volume leading up to time period t,
while \textgreek{D}TV{*}\_t would be a difference of moving averages
of different lengths leading up to time period t. One might make the
averaging window quarterly for example. Finally, the funding weights among
all dApps are scaled to sum to 1.
\begin{itemize}
\item \textbf{The first term} is proportional to $TV_{t}^{*}$, the average
transaction volume generated by the dApp in the recent past. This
is an indicator of the dApp's \textbf{economic activity}, or more
simply the size of its micro-economy.
\item \textbf{The second term} is proportional to $\Delta TV*_{t}/F*_{t}-1$.
The numerator describes the trend in transaction volume \textemdash{}
it is the difference between a more and a less recent average. When
positive, it means that the transaction volume is following an upward
trajectory and vice versa. The denominator is the average funding
amount received by the dApp in the recent past, up to and including
the previous period. So the second term describes how economic activity
is changing relative to past funding. Overall, larger values of this
ratio capture instances where the dApp is fast-growing for each dollar
of funding it has received. This is in fact the spending multiplier
of the funding program, a prime indicator of \textbf{funding efficiency}. 
\item The parameter \textgreek{l} is used to determine the relative importance
of economic activity and funding efficiency. If it isset equal to
1/2 then the two terms would have equal contribution. By decreasing
the value of \textgreek{l}, the protocol can favor more heavily dApps
with larger economies. Conversely, by increasing the value of \textgreek{l}
the protocol can favor dApps that are using funding with high efficiency,
for example by growing fast with little funding, even if they are
smaller in size.
\end{itemize}
The votes on registering and blacklisting a dApp serve to minimize
the risk that the above system is gamed during its infancy. It is
the responsibility of Luna validators to hold dApps accountable for
dishonest behavior and blacklist them if necessary. As the economy
grows and becomes more decentralized, the bar to register and blacklist
an App can be adjusted.

An important advantage of distributing funding in a programmatic way
is that it is simpler, objective, transparent and streamlined compared
to open-ended voting systems. In fact, compared to decentralized voting
systems, it is more predictable, because the inputs used to compute
the funding weights are transparent and slow moving. Furthermore,
this system requires less trust in Luna validators, given that the
only authority they are vested with is determining whether or not
a dApp is honest and makes legitimate use of funding.

Overall, the objective of Terra governance is simple: fund the organizations
and proposals with the highest net impact on the economy. This will
include dApps solving real problems for users, increasing Terra's
adoption and as a result increasing the GDP of the Terra economy. 

\section{Conclusion}

We have presented Terra, a stable digital currency that is designed
to complement both existing fiat and cryptocurrencies as a way to
transact and store value. The protocol adjusts the supply of Terra
in response to changes in demand to keep its price stable. This is
achieved using Luna, the mining token whose countercyclical rewards
are designed to absorb volatility from changing economic cycles. Terra
also achieves efficient adoption by returning seigniorage not invested
in stability back to its users. Its transparent and democratic distribution
mechanism gives dApps the power to attract and retain users by tapping
into Terra's economic growth. 

If Bitcoin\textquoteright s contribution to cryptocurrency was immutability,
and Ethereum expressivity, our value-add will be usability. The potential
applications of Terra are immense. Immediately, we foresee Terra being
used as a medium-of-exchange in online payments, allowing people to
transact freely at a fraction of the fees charged by other payment
methods. As the world starts to become more and more decentralized,
we see Terra being used as a dApp platform where price-stable token
economies are built on Terra. Terra is looking to become the first
usable currency and stability platform on the blockchain, unlocking
the power of decentralization for mainstream users, merchants, and
developers.

\noindent {\small{}\newpage}\textbf{\small{}References }{\small\par}

\noindent {\small{}Liu, Yukun and Tsyvinski, Aleh, Risks and Returns
of Cryptocurrency (August 2018). NBER Working Paper No. w24877. Available
at https://ssrn.com/abstract=3226806. }{\small\par}

\noindent {\small{}Makarov, Igor and Schoar, Antoinette, Trading and
Arbitrage in Cryptocurrency Markets (April 30, 2018). Available at
SSRN: https://ssrn.com/abstract=3171204. }{\small\par}

\noindent {\small{}Kereiakes, Evan, Rationale for Including Multiple
Fiat Currencies in Terra\textquoteright s Peg (November 2018). Available
at https://medium.com/terra-money/rationale-for-including-multiple-fiat-currencies-in-terras-peg-1ea9eae9de2a }{\small\par}

\noindent {\small{}Taylor, John B. (1993). \textquotedbl Discretion
versus Policy Rules in Practice.\textquotedbl{} Carnegie-Rochester
Conference Series on Public Policy. 39: 195\textendash 214. }{\small\par}
\end{document}
%% LyX 2.3.1-1 created this file.  For more info, see http://www.lyx.org/.
%% Do not edit unless you really know what you are doing.
\documentclass[12pt]{article}
\usepackage[LGR,T1]{fontenc}
\usepackage[latin9]{inputenc}
\usepackage{geometry}
\geometry{verbose,tmargin=1in,bmargin=1in,lmargin=1in,rmargin=1in}
\usepackage{color}
\usepackage{amsmath}
\usepackage{amssymb}
\usepackage{graphicx}
\usepackage{setspace}
\doublespacing
\usepackage[unicode=true,
 bookmarks=false,
 breaklinks=false,pdfborder={0 0 1},backref=section,colorlinks=true]
 {hyperref}
\hypersetup{pdftitle={How QE Works: Evidence on the Refinancing Channel},
 linkcolor=black,citecolor=blue,filecolor=magenta,urlcolor=blue}

\makeatletter

%%%%%%%%%%%%%%%%%%%%%%%%%%%%%% LyX specific LaTeX commands.
\DeclareRobustCommand{\greektext}{%
  \fontencoding{LGR}\selectfont\def\encodingdefault{LGR}}
\DeclareRobustCommand{\textgreek}[1]{\leavevmode{\greektext #1}}
\ProvideTextCommand{\~}{LGR}[1]{\char126#1}


%%%%%%%%%%%%%%%%%%%%%%%%%%%%%% User specified LaTeX commands.
\usepackage{color,titling,titlesec}
\usepackage{eurosym}
\usepackage{harvard}\usepackage{amsfonts}
\usepackage{graphicx,pdflscape}
\usepackage{chicago}\setcounter{MaxMatrixCols}{30}
\usepackage[bottom]{footmisc}

% This shrinks the space before and after display formulas
\usepackage{etoolbox}
\apptocmd\normalsize{%
\abovedisplayskip=5pt
%\abovedisplayshortskip=6pt % plus 3pt
 \belowdisplayskip=6pt
 %\belowdisplayshortskip=7pt plus 3pt
}{}{}

% slim white space before title, section/subsection headers
\setlength{\droptitle}{-.75in}
\titlespacing*{\section}{0pt}{.2cm}{0pt}
\titlespacing*{\subsection}{0pt}{.2cm}{0pt}

\makeatother

\begin{document}
\title{Terra Money:\\
Stability and Adoption}
\author{Evan Kereiakes (evan@terra.money) \and Do Kwon (do@terra.money) \and
Marco Di Maggio (marco@terra.money) \and Nicholas Platias (nick@terra.money)}
\date{February 2019}
\maketitle
\begin{center}
{\large{}\vspace{-1.5cm}
}{\large\par}
\par\end{center}
\begin{abstract}
\begin{singlespace}
While many see the benefits of a price-stable cryptocurrency that
combines the best of both fiat and Bitcoin, not many have a clear
plan to get such a currency adopted. Since the value of a currency
as medium of exchange is mainly driven by its network effects, a successful
new digital currency needs to maximize adoption in order to become
useful. We propose a cryptocurrency, Terra, which is both price-stable
and growth-driven. It achieves price-stability via an elastic money
supply enabled by countercyclical mining incentives and transaction
fees. It also uses seigniorage created by its minting operations as
transaction stimulus, thereby facilitating adoption. There is demand
for a decentralized, price-stable money protocol in both fiat and
blockchain economies. If such a protocol succeeds, then it will have
a significant impact as the best use case for cryptocurrencies.
\end{singlespace}
\end{abstract}
\thispagestyle{empty}

\newpage\setcounter{page}{1}

\section{Introduction}

The price-volatility of cryptocurrencies is a well-studied problem
by both academics and market observers (see for instance, Liu and
Tsyvinski, 2018, Makarov and Schoar, 2018). Most cryptocurrencies,
including Bitcoin, have a predetermined issuance schedule that, together
with a strong speculative demand, contributes to wild fluctuations
in price. Bitcoin\textquoteright s extreme price volatility is a major
roadblock towards its adoption as a medium of exchange or store of
value. Intuitively, nobody wants to pay with a currency that has the
potential to double in value in a few days, or wants to be paid in
the currency if its value can significantly decline before the transaction
is settled. The problems are aggravated when the transaction requires
more time, e.g. for deferred payments such as mortgages or employment
contracts, as volatility would severely disadvantage one side of the
contract, making the usage of existing digital currencies in these
settings prohibitively expensive. At the core of how the Terra Protocol
solves these issues is the idea that a cryptocurrency with an elastic
monetary policy would stabilize its price, retaining all the censorship
resistance of Bitcoin, and making it viable for use in everyday transactions.
However, price-stability is not sufficient to get a new currency widely
adopted. Currencies are inherently characterized by strong network
effects: a customer is unlikely to switch over to a new currency unless
a critical mass of merchants are ready to accept it, but at the same
time, merchants have no reason to invest resources and educate staff
to accept a new currency unless there is significant customer demand
for it. For this reason, Bitcoin\textquoteright s adoption in the
payments space has been limited to small businesses whose owners are
personally invested in cryptocurrencies. Our belief is that while
an elastic monetary policy is the solution to the stability problem,
an efficient fiscal policy can drive adoption. Then, the Terra Protocol
also offers strong incentives for users to join the network with an
efficient fiscal spending regime, managed by a Treasury, where multiple
stimulus programs compete for financing. That is, proposals from community
participants will be vetted by the rest of the ecosystem and, when
approved, they will be financed with the objective to increase adoption
and expand the potential use cases. The Terra Protocol with its balance
between fostering stability and adoption represents a meaningful complement
to fiat currencies as means of payment, and store of value. The rest
of the paper is organized as follows. We first discuss the protocol
and how stability is achieved and maintained, through the calibration
of miners\textquoteright{} demand and the use of the native mining
Luna token. We then dig deeper in how countercyclical incentives and
fees are adopted to smooth fluctuations. Second, we discuss how Terra\textquoteright s
fiscal policy can be used as an efficient stimulus to drive adoption. 

\section{Multi-fiat peg monetary policy }

A stable-coin mechanism must answer three key questions: 
\begin{itemize}
\item \textbf{How is price-stability defined?} Stability is a relative concept;
which asset should a stable-coin be pegged to in order to appeal to
the broadest possible audience? 
\item \textbf{How is price-stability measured?} Coin price is exogenous
to the Terra blockchain, and an efficient, corruption-resistant price
feed is necessary for the system to function properly. 
\item \textbf{How is price-stability achieved?} When coin price has deviated
from the target, the system needs a way to apply pressures to the
market to bring price back to the target. 
\end{itemize}
This section will specify Terra\textquoteright s answers to the above
questions in detail. 

\subsection{Defining stability against regional fiat currencies }

The existential objective of a stable-coin is to retain its purchasing
power. Given that most goods and services are consumed domestically,
it is important to create crypto-currencies that track the value of
local fiat currencies. Though the US Dollar dominates international
trade and forex operations, to the average consumer the dollar exhibits
unacceptable volatility against her choice unit of account. 

Recognizing strong regionalities in money, Terra aims to be a family
of cryptocurrencies that are each pegged to the world's major currencies.
Close to genesis, the protocol will issue Terra currencies pegged
to USD, EUR, CNY, JPY, GBP, KRW, and the IMF SDR. Over time, more
currencies will be added to the list by user voting. TerraSDR will
be the flagship currency of this family, given that it exhibits the
lowest volatility against any one fiat currency (Kereiakes, 2018).
TerraSDR be the currency in which payment of transaction fees, disbursements
of miner rewards and fiscal spending will be denominated.

\subsection{Measuring stability with miner oracles }

Since the price of Terra currencies in secondary markets is exogenous
to the blockchain, the system must rely on a decentralized price oracle
to estimate the true exchange rate. We define the mechanism for the
price oracle as the following: 
\begin{itemize}
\item For any Terra sub-currency in the set of currencies C = \{TerraKRW,
TerraUSD, TerraSDR... \} miners submit a vote for what they believe
to be the current exchange rate between the sub-currency and its target
fiat asset. To prevent front-running, voters submit a hash of the
price value instead of the price itself. 
\item Every n blocks the vote is tallied by having voters submit a solution
to the vote hash. The weighted median of the votes is taken for both
the target and observed exchange rates as the true rates. 
\item Some amount of Terra is rewarded to those who voted within 1 standard
deviation of the elected median. Those who voted outside may be punished
via slashing of their stakes. The ratio of those that are punished
and rewarded may be calibrated by the system every vote to ensure
that a sufficiently large portion of the miners vote. 
\end{itemize}
Several issues have been raised in implementing decentralized oracles,
but chief among them is the possibility for voters to profit by coordinating
on a false price vote. Limiting the vote to a specific subset of users
with strong vested interest in the system, the miners, can vastly
decrease the odds of such a coordination. A successful coordination
event on the price oracle would result in a much higher loss in the
value of the miner stakes than any potential gains of a successful
coordination. 

\subsection{Achieving stability through countercyclical mining }

Once the system has detected that the price of a Terra currency has
deviated from its peg, it must apply pressures to normalize the price.
Like any other market, the Terra money market follows simple rules
of supply and demand for a pegged currency. That is: 
\begin{itemize}
\item Contracting money supply, all conditions held equal, will result in
higher relative price levels. That is, when price levels are falling
below the target, reducing money supply sufficiently will return price
levels to normalcy. 
\item Expanding money supply, all conditions held equal, will result in
lower relative price levels. That is, when price levels are rising
above the target, increasing money supply sufficiently will return
price levels to normalcy. 
\end{itemize}
Of course, contracting the supply of money isn\textquoteright t free;
like any other asset, money needs to be bought from the market. Central
banks and governments shoulder contractionary costs for pegged fiat
systems through a variety of mechanisms including intervention, the
issuance of bonds and short-term instruments thus incurring costs
of interest, and hiking of money market rates and reserve ratio requirements
thus losing revenue. Put in a different way, central banks and governments
absorb the volatility of the pegged currencies they issue. 

Analogously, Terra miners absorb volatility in Terra supply. 
\begin{itemize}
\item \textbf{In the short term, miners absorb Terra contraction costs}
through mining power dilution. During a contraction, the system mints
and auctions more mining power to buy back and burn Terra. This effectively
contracts the supply of Terra until its price has returned to the
peg, and temporarily results in mining power dilution. 
\item \textbf{In the mid to long term, miners are compensated with increased
mining rewards}. First, the system continues to buy back mining power
until a fixed target supply is reached, thereby creating long-run
dependability in available mining power. Second, the system increases
mining rewards, which will be explained in more detail in a later
section. 
\end{itemize}
In summary, miners bear the costs of Terra volatility in the short
term, while being compensated for it in the long-term. Compared to
ordinary users, miners have a long-term vested interest in the stability
of the system, with invested infrastructure, trained staff and business
models with high switching cost. The remainder of this section will
discuss how the system forwards short-term volatility and creates
long-term incentives for Terra miners. 

\subsection{Miners absorb short-term Terra volatility }

The Terra Protocol runs on a Proof of Stake (PoS) blockchain, where
miners need to stake a native cryptocurrency Luna to mine Terra transactions.
At every block period, the protocol elects among the set of staked
miners a block producer, which is entrusted with the work required
to produce the next block by aggregating transactions, achieving consensus
among miners, and ensuring that messages are distributed properly
in a short timeframe with high fault tolerance. 

The block producer election is weighted by the size of the active
miner\textquoteright s Luna stake. Therefore, \textbf{Luna represents
mining power in the Terra network.} Similar to how a Bitcoin miner\textquoteright s
hash power represents a pro-rata odds of generating Bitcoin blocks,
the Luna stake represents pro-rata odds of generating Terra blocks. 

Luna also serves as the most immediate defense against Terra price
fluctuations. The system uses Luna to make the price for Terra by
agreeing to be counter-party to anyone looking to swap Terra and Luna
at Terra's target exchange rate. More concretely: 
\begin{itemize}
\item When TerraSDR's price < 1 SDR, users and arbitragers can send 1 TerraSDR
to the system and receive 1 SDR's worth of Luna. 
\item When TerraSDR's price > 1 SDR, users and arbitragers can send 1 SDR's
worth of Luna to the system and receive 1 TerraSDR. 
\end{itemize}
The system's willingness to respect the target exchange rate irrespective
of market conditions keeps the market exchange rate of Terra at a
tight band around the target exchange rate. 

The system finances Terra price making via Luna: 
\begin{itemize}
\item To buy 1 TerraSDR, the protocol mints and sells Luna worth 1 SDR 
\item By selling 1 TerraSDR, the protocol earns Luna worth 1 SDR 
\end{itemize}
As Luna is minted to match Terra offers, volatility is moved from
Terra price to Luna supply. If unmitigated, this Luna dilution presents
a problem for miners; their Luna stakes are worth a smaller portion
of total available mining power post-contraction. Therefore, the system
burns a portion of the Luna it has earned during expansions until
Luna supply has reached its 1 billion equilibrium issuance. Therefore,
Luna can have steady demand as a token with pro-rata rights to Terra
mining over the long term. The next section discusses how the system
offers countercyclical mining incentives to keep the market for mining
and demand for Luna long-term stable through volatile macroeconomic
cycles. 

\subsection{Countercyclical Mining Rewards }

Our objective is to counteract fluctuations in the value of mining
Terra by calibrating mining rewards to be countercyclical. The main
intuition behind a countercyclical policy is that it attempts to counteract
economic cycles by increasing mining rewards during recessions and
decreasing mining rewards during booms. The protocol has two levers
at its disposal to calibrate mining rewards: transaction fees, and
the proportion of seigniorage that gets allocated to miners. 
\begin{itemize}
\item Transaction fees: The protocol levies a small fee from every Terra
transaction to reward miners. Fees default to 0.1\% but may vary over
time to smooth out mining rewards. If mining rewards are declining,
an increase in fees can reverse that trend. Conversely, high mining
rewards give the protocol leeway to bring fees down. Fees are capped
at 2\%, so they are restricted to a range that does not exceed the
fees paid to traditional payment processors. 
\item Seigniorage: Users can mint Terra by paying the system Luna. This
Luna earned by the system is seigniorage, the value of newly minted
currency minus the cost of issuance. The system burns a portion of
seigniorage, which makes mining power scarcer and reduces mining competition.
The remaining portion of seigniorage goes to the Treasury to finance
fiscal stimulus. The system can calibrate the allocation of seigniorage
between those two destinations to impact mining reward profiles. 
\end{itemize}
We use two key macroeconomic indicators as inputs for controlling
mining rewards: money supply and transaction volume. Those two indicators
are important signals for the performance of the economy. Looking
at the equation for mining rewards: when the economy underperforms
relative to recent history (money supply and transaction volume have
decreased) a higher proportion of seigniorage is allocated to mining
rewards, and transaction fees increase; conversely, when the economy
outperforms recent history (money supply and transaction volume have
increased) a lower proportion of seigniorage is allocated to mining
rewards, and transaction fees decrease. 

Phrasing the above more formally, both the proportion of seigniorage
that gets allocated to mining rewards wt and the adjustment in transaction
fees ft are controlled over time as follows:

\[
w_{t+1}=w_{t}+\beta\left(\frac{M_{t}}{M_{t}^{*}}-1\right)+\gamma\left(\frac{TV_{t}}{TV_{t}^{*}}-1\right)
\]

\[
f_{t+1}=f_{t}+\kappa\left(\frac{M_{t}}{M_{t}^{*}}-1\right)+\lambda\left(\frac{TV_{t}}{TV_{t}^{*}}-1\right)
\]

In the above $M_{t}$ is Terra money supply at time $t$, $M_{t}^{*}$
is the historical moving average of money supply over the previous
quarter, $TV_{t}$ is Terra transaction volume at time $t$ and $TV_{t}^{*}$
is correspondingly the historical moving average of transaction volume
over the previous quarter. The parameters \textgreek{b}, \textgreek{g},
\textgreek{k} and \textgreek{l} are all negative real numbers in the
range {[}-1, 0) and will be calibrated to produce responses that are
gradual but effective. Indicative values that have worked well in
our simulations are between -0.5 and -1 for \textgreek{b}, \textgreek{g}
and between -0.005 and -0.01 for \textgreek{k}, \textgreek{l} respectively.
Both wt and ft are restricted within the range imposed by the protocol
(between 0\% and 90\% for wt, and between 0.01\% and 2\% for ft). 

To see how this might work in practice: say that money supply is 10\%
higher than quarterly average and transaction volume is 20\% higher
than quarterly average. Let \textgreek{b}, \textgreek{g} be -0.5 and
\textgreek{k}, \textgreek{l} be -0.01 respectively. The seigniorage
allocation weight to mining rewards would decrease by 15\% and transaction
fees would decrease by 0.3\% (both on an absolute basis). These are
reasonable adjustments that allocate proportionally more capital to
the Treasury and ease the fee burden on users in response to strong
performance of the economy.
\begin{center}
\includegraphics[scale=1.25]{graph_tvgap}
\par\end{center}

Alternatively, as the graph shows, when the gap widens, i.e. as the
economy shrinks, the fees will increase from the example starting
point of 0.1\% to the maximum of 2\% that will still make Terra more
convenient than Visa and Mastercard. 

The rule we have outlined for making mining rewards countercyclical
is simple, intuitive and easily programmable. It takes inspiration
from Taylor\textquoteright s Rule (Taylor, 1993), utilized by monetary
authorities banks to help frame the level of nominal interest rates
as a function of inflation and output. Similarly, exactly as a central
bank, the protocol observes the health of the economy, in our case
the money supply and transaction volume, and adjusts its main levers
to ensure the sustainability of the economy.

\section{Growth-driven fiscal policy}

National governments use expansionary fiscal spending with the objective
of stimulating growth. On the balance, the hope of fiscal spending
is that the economic activity instigated by the original spending
results in a feedback loop that grows the economy more than the amount
of money spent in the initial stimulus. This concept is captured by
the spending multiplier \textemdash{} how many dollars of economic
activity does one dollar of fiscal spending generate? The spending
multiplier increases with the marginal propensity to consume, meaning
that the effectiveness of the expansionary stimulus is directly related
to how likely economic agents are to increase their spending. 

In a previous section, we discussed how Terra seigniorage is directed
to both miner rewards and the Treasury. At this point, it is worth
describing how exactly the Treasury implements Terra's fiscal spending
policy, with its core mandate being stimulating Terra's growth while
ensuring its stability. In this manner, Terra achieves greater efficiency
by returning seigniorage not allocated for stability back to its users.

The Treasury main focus is the allocation of resources derived from
seigniorage to decentralized application (dApp). To receive seigniorage
from the Treasury, a dApp needs to register for consideration as an
entity that operates on the Terra network. dApps are eligible for
funding depending on their economic activity and use of funding. A
dApp registers a wallet with the network that is used to track economic
activity. Transactions that go through the wallet count towards the
dApp's transaction volume.

The funding procedure for a dApp works as follows: 
\begin{itemize}
\item A dApp applies for an account with the Treasury; the application includes
metadata such as the Title, a url leading to a detailed page regarding
the use of funding, the wallet address of the applicant, as well as
auditing and governance procedures. 
\item At regular voting intervals, Luna validators vote to accept or reject
new dApp applications for Treasury accounts. The net number of votes
(yes votes minus no votes) needs to exceed 1/3 of total available
validator power for an application to be accepted.
\item Luna validators may only exercise control over which dApps can open
accounts with the Treasury. The funding itself is determined programmatically
for each funding period by a weight that is assigned to each dApp.
This allows the Treasury to prioritize dApps that earn the most funding. 
\item At each voting session, Luna validators have the right to request
that a dApp be blacklisted, for example because it behaves dishonestly
or fails to account for its use of Treasury funds. Again, the net
number of votes (yes votes minus no votes) needs to exceed 1/3 of
total available validator power for the blacklist to be enforced.
A blacklisted dApp loses access to its Treasury account and is no
longer eligible for funding.
\end{itemize}
The motivation behind assigning funding weights to dApps is to maximize
the impact of the stimulus on the economy by rewarding the dApps that
are more likely to have a positive effect on the economy. The Treasury
uses two criteria for allocating spending: (1) \textbf{robust economic
activity} and (2) \textbf{efficient use of funding}. dApps with a
strong track record of adoption receive support for their continued
success, and dApps that have grown relative to their funding are rewarded
with more seigniorage, as they have a successful track record of efficiently
using their resources. 

Those two criteria are combined into a single weight which determines
the relative funding that dApps receive from the aggregate funding
pool. For instance, a dApp with a weight of 2 would receive twice
the amount of funding of a dApp with a weight of 1.

We lay out the funding weight equation, followed by a detailed explanation
of all the parts: For a time period t, let TV\_t be a dApp's transaction
volume and F\_t be the Treasury funding received. Then we determine
the funding weight $w_{t}$ for the period as follows:

\[
w_{t}=\left(1-\lambda\right)TV_{t}^{*}+\lambda\frac{\Delta TV_{t}^{*}}{F_{t-1}^{*}}
\]

The notation {*} denotes a moving average, so TV{*}\_t would be the
moving average of transaction volume leading up to time period t,
while \textgreek{D}TV{*}\_t would be a difference of moving averages
of different lengths leading up to time period t. One might make the
averaging window quarterly for example. The parameters \textgreek{k},
\textgreek{l} are between 0 and 1 and determine the relative importance
of the two terms being summed. Finally, the funding weights among
all dApps are scaled to sum to 1.
\begin{itemize}
\item \textbf{The first term} is proportional to $TV_{t}^{*}$, the average
transaction volume generated by the dApp in the recent past. This
is an indicator of the dApp's \textbf{economic activity}, or more
simply the size of its micro-economy.
\item \textbf{The second term} is proportional to $\Delta TV*_{t}/F*_{t}-1$.
The numerator describes the trend in transaction volume \textemdash{}
it is the difference between a more and a less recent average. When
positive, it means that the transaction volume is following an upward
trajectory and vice versa. The denominator is the average funding
amount received by the dApp in the recent past, up to and including
the previous period. So the second term describes how economic activity
is changing relative to past funding. Overall, larger values of this
ratio capture instances where the dApp is fast-growing for each dollar
of funding it has received. This is in fact the spending multiplier
of the funding program, a prime indicator of \textbf{funding efficiency}. 
\item The parameter \textgreek{l} is used to determine the relative importance
of economic activity and funding efficiency. If it isset equal to
1/2 then the two terms would have equal contribution. By decreasing
the value of \textgreek{l}, the protocol can favor more heavily dApps
with larger economies. Conversely, by increasing the value of \textgreek{l}
the protocol can favor dApps that are using funding with high efficiency,
for example by growing fast with little funding, even if they are
smaller in size.
\end{itemize}
The votes on registering and blacklisting a dApp serve to minimize
the risk that the above system is gamed during its infancy. It is
the responsibility of Luna validators to hold dApps accountable for
dishonest behavior and blacklist them if necessary. As the economy
grows and becomes more decentralized, the bar to register and blacklist
an App can be adjusted.

An important advantage of distributing funding in a programmatic way
is that it is simpler, objective, transparent and streamlined compared
to open-ended voting systems. In fact, compared to decentralized voting
systems, it is more predictable, because the inputs used to compute
the funding weights are transparent and slow moving. Furthermore,
this system requires less trust in Luna validators, given that the
only authority they are vested with is determining whether or not
a dApp is honest and makes legitimate use of funding.

Overall, the objective of Terra governance is simple: fund the organizations
and proposals with the highest net impact on the economy. This will
include dApps solving real problems for users, increasing Terra's
adoption and as a result increasing the GDP of the Terra economy. 

\section{Conclusion}

We have presented Terra, a stable digital currency that is designed
to complement both existing fiat and cryptocurrencies as a way to
transact and store value. The protocol adjusts the supply of Terra
in response to changes in demand to keep its price stable. This is
achieved using Luna, the mining token whose countercyclical rewards
are designed to absorb volatility from changing economic cycles. Terra
also achieves efficient adoption by returning seigniorage not invested
in stability back to its users. Its transparent and democratic distribution
mechanism gives dApps the power to attract and retain users by tapping
into Terra's economic growth. 

If Bitcoin\textquoteright s contribution to cryptocurrency was immutability,
and Ethereum expressivity, our value-add will be usability. The potential
applications of Terra are immense. Immediately, we foresee Terra being
used as a medium-of-exchange in online payments, allowing people to
transact freely at a fraction of the fees charged by other payment
methods. As the world starts to become more and more decentralized,
we see Terra being used as a dApp platform where price-stable token
economies are built on Terra. Terra is looking to become the first
usable currency and stability platform on the blockchain, unlocking
the power of decentralization for mainstream users, merchants, and
developers.

\noindent {\small{}\newpage}\textbf{\small{}References }{\small\par}

\noindent {\small{}Liu, Yukun and Tsyvinski, Aleh, Risks and Returns
of Cryptocurrency (August 2018). NBER Working Paper No. w24877. Available
at https://ssrn.com/abstract=3226806. }{\small\par}

\noindent {\small{}Makarov, Igor and Schoar, Antoinette, Trading and
Arbitrage in Cryptocurrency Markets (April 30, 2018). Available at
SSRN: https://ssrn.com/abstract=3171204. }{\small\par}

\noindent {\small{}Kereiakes, Evan, Rationale for Including Multiple
Fiat Currencies in Terra\textquoteright s Peg (November 2018). Available
at https://medium.com/terra-money/rationale-for-including-multiple-fiat-currencies-in-terras-peg-1ea9eae9de2a }{\small\par}

\noindent {\small{}Taylor, John B. (1993). \textquotedbl Discretion
versus Policy Rules in Practice.\textquotedbl{} Carnegie-Rochester
Conference Series on Public Policy. 39: 195\textendash 214. }{\small\par}
\end{document}
